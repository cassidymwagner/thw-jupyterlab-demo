\input{top-apj}
\usepackage{graphicx}
\usepackage{amsmath}
\usepackage{courier}

% Headers for odd and even pages, respectively:
\shorttitle{Wagner}
% If more than two authors, use {\em et al.}
\shortauthors{Wagner}


\begin{document}

\title{Constraining the star formation contribution to \\ 
       the infrared luminosity in quasars}


%% AUTHOR/INSTITUTIONS FOR AASTEX6.1:
\author{Cassidy Wagner}

\affil{Homer L. Dodge Department of Physics and Astronomy, University of Oklahoma, 440 West Brooks Street, 
       Norman, OK 73019, USA} 
\begin{abstract}

All quasars are fundamentally similar, powered by accretion of matter onto a supermassive black hole. A substantial fraction 
of quasars exhibit broad absorption lines in their spectra, indicative of high-velocity wind outflow. These winds have been 
theorized to be responsible for feedback in galaxy evolution. \cite{farrah2012} claim an anticorrelation between 
outflow strength and star formation contribution, concluding that windy outflows quench star formation.

Although \cite{farrah2012} supports the claim that quasar winds play a role in feedback, we find that their methods are limited 
in some areas. The outflow strength is insufficiently quantified through the absorption strength parameter which is 
only loosely related to the outflow strength. We also believe some of their models may be outdated and inaccurate. Several 
fits show an unobserved level of polycyclic aromatic hydrocarbons, indicating a flawed starburst model. Additionally, shorter 
wavelength regions than those used in the \cite{farrah2012} analysis further constrain the reddening component.

We use \texttt{emcee} (\citealp{foremanmackey2013}), a Python implementation of the Affine Invariant MCMC Ensemble sampler, to fit the 
spectral energy distributions of the \cite{farrah2012} objects with alternative component models. We focus on constraining 
the starburst and reddening components of the objects with starburst templates extracted from the Bayesian tool, AGNfitter 
(\citealp{rivera2016}), along with 
templates for the host galaxy (\citealp{fioc1997}) and quasar (\citealp{lyu2017feb}) to fully fit the objects. As our 
reddening template of choice goes to shorter wavelengths, 
we obtain additional photometric data at this region to aid in the reddening constraint.  

\end{abstract}

\section{INTRODUCTION}
\label{introduction}

It has been observed that galaxies and their central supermassive black holes grow in a correlated fashion, implying that 
as the galaxy itself grows, so too does the black hole. However, the black hole makes up an incredibly small fraction of its host galaxy, 
pushing research toward investigating this relationship. To understand quasar and galaxy evolution, the notion of feedback has come into 
the forefront of quasar research.

A promising explanation for quasar feedback are wind outflows, high-velocity outflows of gas from the central engine. It has been 
theorized that these winds may go into the galaxy and disrupt regions of dense star formation, implying the black hole plays a role 
in influencing the evolution of the host galaxy. \cite{farrah2012} claim to have found evidence that there exists an anticorrelation 
between the strength 
of these wind outflows and the star formation contribution in the infrared (IR) luminosity. 

The outflows are identified by broad absorption lines in quasar spectra, while the star formation contribution can be seen in the 
spectral energy distribution (SED) of the quasar. \cite{farrah2012} focused on constraining the outflow strength and 
star formation contribution to claim that when the outflow strength was highest for an object, the star formation luminosity would be lower. 
The scope of this project is to contribute to a similar analysis using different methods. 

In this paper we use \texttt{emcee} to fit quasar SEDs for the purpose of constraining reddening and the starburst component. We begin by 
discussing our sample selection and justifying our choice of object to focus on for this paper. We then provide the components 
used to construct the SED fits and elaborate on how each component was developed. We give a brief introduction to MCMC and \texttt{emcee} before 
presenting our results and commenting on the successes and challenges throughout this project. 

\section{Data}
\label{sec:data}

\subsection{Sample selection}
\label{sec:sample}

Our sample has been extracted from \cite{farrah2012} who chose the objects based explicitly on their rest-frame UV spectral 
features and if the objects had optical spectra. Their sample contains 31 objects such that the sample 
is a random subset of the FeLoBAL quasar population between $0.8 < z < 1.8$ with no selection on IR luminosity (\citealp{farrah2012}). 
Additionally, the spectrum of each object contains the Mg II broad absorption line. 

\cite{farrah2012} uses Spitzer MIPS photometry, WISE photometry, and 2MASS photometry. We use their magnitudes for MIPS, but retrieve the 
WISE photometry from \cite{wright2010}. We have also reduced the \cite{farrah2012}  sample of 31 down to 25 as some objects 
were misclassified as BAL quasars. 

\subsection{Additional data collection and preparation}
\label{sec:sample}

Many objects in the \cite{farrah2012} sample are faint at shorter wavelengths, resulting in larger errors. The JHK photometry 
is all from 2MASS, a shallow survey that picks up the brightest objects. 2MASS can achieve SNR=10 for objects of magnitude 15.8,15.1, 
and 14.3 for the J,H, and K bands, respectively (\citealp{skrutskie2006}). Our objects are fainter than this, therefore 
an adequate SNR for our purposes is not 
achieved with 2MASS. Thus we replace the 2MASS photometry with UKIDSS photometry, a survey with a depth of three magnitudes deeper than 
2MASS, to more effectively constrain the fit (\citealp{warren2007}). 
The \cite{farrah2012} analysis does not include shorter wavelength photometry such as SDSS \textit{ugriz}, therefore we included 
this region to better constrain the reddening and quasar components. 

We have obtained additional photometry from observations with NICFPS on APO's 2.5 meter as well as with TIFKAM on MDM's 2.4 meter, 
not yet included in this paper and awaiting reduction. We obtained 
H-band photometry for some objects from the MDM observations as well as H-band and K-band photometry with APO. Since these 
data are not included in this paper, our sample of 25 is reduced to 14 to ensure that our JHK photometry is all from the more sensitive UKIDSS 
catalog.

\section{SED Components and Parameters}
\label{sec:models}

We use four parameters to constrain the SED fits for these objects: the amplitudes of the quasar component, made up of the radiation from 
the accretion disk and the torus, the star formation component, and the host galaxy component, as well as a parameter for the 
Small Magellanic Cloud (SMC) reddening law (\citealp{prevot1984}). The SED model components are shown in Fig.~\ref{fig:comps} 
and all parameters are discussed in detail below. 

\begin{figure}[t]
  \centering
  \includegraphics[width=\linewidth]{figs/all_comps_reg.png}
  \caption{Three components used to construct the SED fits. The blue line corresponds to  
           the host galaxy component from \cite{fioc1997}. The orange line is the quasar component from 
           \cite{lyu2017feb}, made up of the radiation from both the accretion disk and the torus. 
           The green line is a single starburst component, one of the templates in the 
           \cite{dalehelou2002} library.}
           \label{fig:comps}
\end{figure}

\subsection{Quasar component}
\label{sec:bbb}
As matter spirals into the black hole, a fraction of its rest energy can be released and converted into heat and radiation which 
presents makes up the shorter wavelength portion of the entire quasar component. As the radiation from the accretion disk comes 
into contact with the torus, the dust in the torus absorbs and reprocesses the radiation, emitting in the infrared.

The model for the quasar, shown in Fig.~\ref{fig:comps}, comes from \cite{lyu2017feb} who construct three separate models for the quasar based on 
dust deficiencies. They begin by using the \cite{elvis1994} template to fit the SEDs of 87 Palomer-Green quasars, a quasar 
SED template based on a sample of optically-selected and radio-selected quasars with strong X-ray emission that are optically blue.
However, \cite{lyu2017feb} finds that ~40\% of the PG SEDs are not fit well with the Elvis template. 
They separate the unfit objects into two groups: 
hot-dust-deficient (HDD) and warm-dust-deficient (WDD), which make up ~15\%-23\% and ~14\%-17\% of the sample, respectively. 
Finally they construct composite templates from the two 
groups and provide models for WDD and HDD quasars. The normal Elvis, WDD, and HDD templates are shown in Fig.~\ref{fig:lyu}.
  
\begin{figure}[t]
  \centering
  \includegraphics[width=\linewidth]{figs/lyu_temps.png}
  \caption{Three quasar templates constructed in \cite{lyu2017feb} The blue line is the normal template from \cite{elvis1994}, 
           without dust deficiencies taken into consideration. The orange line is the WDD model while the green line is the 
           HDD model. All models are normalized at 1.25$\mu$m.
         }
         \label{fig:lyu}
\end{figure}
  
\subsection{Star formation component}
\label{sec:star}

As young stars evolve, they emit UV-visible light, which is then absorbed by dust left over from previous 
generations of star formation and reradiated in the infrared. 

A total of 169 starburst templates were extracted from \cite{rivera2016} who constructed them from separate libraries 
in \cite{charyelbaz2001} and \cite{dalehelou2002}. The starburst template used in this paper comes from the 
\cite{dalehelou2002} library, although we discuss later how a range of templates should be used to explore the fitting process.
\cite{charyelbaz2001} constructed 105 starburst templates based on four SEDs of prototypical starburst galaxies. The templates 
were derived using the \cite{silva1998} models. To obtain the 105 templates from the original four, \cite{dalehelou2001} 
interpolated between the galaxies for a more diverse set.

The 64 \cite{dalehelou2002} templates are updated versions of the original templates in \cite{dalehelou2001}. These 
templates involved three components, large dust grains in thermal equilibrium, small grains semistochastically heated, and stochastically 
heated polycyclic aromatic hydrocarbons. The models are based on IRAS/ISO observations of 69 normal star-forming galaxies at a range 
of 3--100$\mu$m (\cite{rivera2016}). To obtain the 64 templates, the original models are improved upon with longer 
wavelength observations (\citealp{dalehelou2002}). The single starburst template used in our analysis is shown in Fig.~\ref{fig:comps}. 

\subsection{Host galaxy component}
\label{sec:galaxy}

The radiation from the host galaxy comes from starlight, although this is often overpowered by the luminosity of 
the quasar.

The host galaxy component is from \cite{fioc1997} who introduce their spectral evolution model of galaxies, PEGASE. 
The component shown in Fig.~\ref{fig:comps} is composed of a 13 Gyr single-burst galaxy population constructed with 
the PEGASE models.

\subsection{SMC reddening parameter}
\label{sec:smc}

To correct for reddening within the host galaxy, we use a reddening model based on empirical analysis of the Small Magellanic Cloud 
(\citealp{prevot1984}). The reddening correction is applied to the sum of all three components by the following equation:

\begin{equation}
  \label{eq:red}
  F_{\lambda,\mathrm{obs}} = F_{\lambda,\mathrm{int}} \times 10^{-0.4A_{\lambda}},
\end{equation}

where $F_{\lambda,\mathrm{int}}$ is the flux prior to the correction and $A_{\lambda}$ contains the correction. 

Reddening affects the shorter wavelength region of the spectrum more than the longer wavelength region, thus implementing 
the reddening correction does not affect the fit near the starburst component very much. However, constraining the outflow 
strength is accomplished through fitting the visible spectra of these objects, many of which are very reddened and require 
more accurate reddening values. 

\section{MCMC Fitting}
\label{sec:fitting}

For our purposes in fitting the SEDs, we have a 4-dimensional parameter space corresponding to the parameters discussed in \S~\ref{sec:models}. 
Therefore we find it useful to fit these objects with a Markov Chain Monte Carlo algorithm as opposed to altering the parameters individually 
and attempting to minimize the error between the data and the model fit. 

\subsection{MCMC algorithm}
\label{sec:mcmc}

MCMC is a Bayesian approach to fitting with a posterior probability given as

\begin{equation}
  \label{eq:linearfunc}
  P(B|A) \sim P(A|B)P(B),
\end{equation} 

where $P(B|A)$ is the posterior probability, $P(A|B)$ is the likelihood, and $P(B)$ is the prior.  
 
A simple MCMC algorithm, such as Metropolis-Hastings,  will start at a random position in parameter space and evaluate the likelihood at 
that position. It will then choose a new position based on a transition probability and evaluate the likelihood at the new position. 
If the new-position likelihood indicates that this position in parameter space will result in a better fit, the new position 
is accepted. If the new  position appears to be a worse fit, there is still some probability that the new position is accepted 
anyway in order to evenly sample parameter space as well as to avoid clustering in local minima. This process is repeated 
iteratively until convergence is reached. The Monte Carlo part of the algorithm describes the repeated random sampling while 
the Markov Chain part describes how the algorithm will fairly sample parameter space.

\subsection{Emcee}
\label{sec:emcee}

A particular disadvantage of the Metropolis-Hastings method is that its transition probability is correlated with the shape of 
parameter space. The algorithm requires that each new position is calculated from a transition function of fixed width, meaning that 
a step in any dimension of parameter space is constrained by whichever dimension requires the smallest deviations from the initial 
position. Therefore, if the algorithm were attempting to sample an ellipse, its step size could not exceed the appropriate step 
size when sampling the semi-minor axis of parameter space. With smaller step sizes, sampling parameter space becomes more 
computationally expensive. 

Emcee Hammer is a pythonic implementation of an affine-invariant MCMC Ensemble sampler which solves the problem 
described above (\citealp{foremanmackey2013}). The \texttt{emcee} algorithm employs an affine transition function 
to draw the next step, implying that although the 
function transforms, it will not change the posterior probability distribution, thus it is affine-invariant. The transition function has 
the form 

\begin{equation}
  \label{eq:linearfunc}
  y = Ax + b,
\end{equation}

which has the following property: 

\begin{equation}
  \label{eq:linearfuncpi}
  \pi_{A,b}(y) = \pi_{A,b}(Ax+b) \propto \pi (x),
\end{equation} 

where $\pi$ corresponds to the posterior probability distribution.
Therefore, with \texttt{emcee} we achieve a fair sample of parameter space and convergence much more quickly.

The \texttt{emcee} algorithm initializes a set of walkers permitted to take steps within parameter space. The next step of an individual 
walker is generated based on the positions of the other walkers. If the other walkers happen to be reaching convergence in a region of 
parameter space, then an individual walker may make the decision to move toward the other walkers and converge as well. The walkers 
then converge to a probability distribution for the particular fit parameter, resulting in a confidence region for that parameter. 

\subsection{Running emcee}
\label{sec:run}

To set up our \texttt{emcee} we used $\chi^2$ for the log likelihood and a small region of parameter space around approximated good 
parameters for the prior. We initialized 100 walkers and ran 500 simulations, choosing to remove the first 250 simulations as the 
burn-in while the walkers sampled the space. We performed \texttt{emcee} fits on all 14 objects, shown and discussed 
in \S~\ref{sec:results}. 

\section{Results and Discussion}
\label{sec:results}

We calculated preliminary fits for all 16 objects to constrain the SED components as well as the SMC reddening parameter. The 
resulting fits are shown in Figs.~\ref{fig:fit1} and \ref{fig:fit2}.

\begin{figure*}
  \centering
  \includegraphics[width=\linewidth]{figs/1-change.png}
  \caption{Fits of the first eight objects, corresponding to the objects listed in Tab.~\ref{tab:starlum}. Photometry is in black, 
           photometry that was not used in the fit is shown as red circles, 
           the host galaxy in blue, the quasar in orange, the starburst in green, and the final fit in red. The 99\% confidence 
           region for the fit is shown in gray.} 
           \label{fig:fit1}
\end{figure*}

\begin{figure*}
  \centering
  \includegraphics[width=\linewidth]{figs/2-change.png}
  \caption{Fits of the last six objects, corresponding to the objects listed in Tab.~\ref{tab:starlum}. The details are the same as 
           Fig.~\ref{fig:fit1}.} 
           \label{fig:fit2}
\end{figure*}


\subsection{SED fits}
\label{sec:fits}

We used \texttt{emcee} to construct preliminary fits of the 16 SEDs. Each 
chain converged relatively quickly, allowing for the extraction of the necessary fitting parameters. However, due to low quality data, 
some objects were more difficult to fit, requiring more attention. 

Most objects exhibited high or anomalous reddening in their SDSS spectra, greatly reducing the flux for the shorter wavelength 
photometry. To account for this, we did not use \textit{ugr} photometry to constrain the fit for some objects, 
varying which bands we excluded through 
overlaying the SDSS filter curves onto the spectra and determining where the reddening and contaminating absorption lines were most severe. 
This exclusion improved each fit, 
especially for the objects with such harsh reddening that their short wavelength photometry dropped off by two orders of magnitude. 

Some of the objects appeared to contain strong H-$\alpha$ lines, indicating a possible photometric contamination. Therefore 
we removed J- and H- band photometry from some of these objects, depending on where the H-$\alpha$ lines fell in the observed frame. 
We also removed the longest wavelength MIPS photometry for some objects as the \texttt{emcee} analysis could not evaluate the fit. 
The reason for this is presently unclear, therefore more testing is necessary such that we can return to using this point given 
that it constrains the star formation at longer wavelengths.

Although these changes did improve the fits overall, some of the fits are still poorly constrained at shorter wavelengths, which 
factors into the final calculation of the starburst luminosity. 

\subsection{Calculating starburst luminosity}
\label{sec:starlum}

To compare our results to the results of \cite{farrah2012}, we calculated the luminosity of the starburst 
component for each object. We integrated $F_{\lambda}$ over 1-1000 $\mu m$ in the rest frame to remain consistent 
with Farrah's analysis. We then performed a simple conversion from integrated flux to integrated luminosity by multiplying 
the flux by $4\pi D^2$. $D$, the luminosity distance, is calculated with Ned Wright's Javascript Cosmological Calculator, inputting the 
appropriate redshift (\citealp{wright2006}). The resulting starburst luminosities are shown in Tab.~\ref{tab:starlum}.

\begin{table}
  \centering
  \caption{Log luminosities of the starburst contributions to the SED of all objects.}
  \label{tab:starlum}
  \begin{tabular}{lcccr}
    \hline \hline
    Name & IR L of Starburst ($\log L_{\odot}$) \\
    
    \hline
    SDSS J011117.34+142653.6 & $11.93^{+0.09}_{-1.0}$ \\
    SDSS J030000.57+004828.0 & $11.12^{+0.1}_{-1.9}$ \\
    SDSS J033810.84+005617.7 & $11.92^{+0.1}_{-1.3}$ \\
    SDSS J100605.66+051349.0 & $12.41^{+0.2}_{-0.5}$ \\
    SDSS J101927.36+022521.4 & $11.89^{+0.6}_{-0.1}$ \\
    SDSS J112526.12+002901.3 & $11.62^{+0.3}_{-0.9}$ \\
    SDSS J112828.31+011337.9 & $11.29^{+0.4}_{-0.6}$ \\
    SDSS J112901.71+050617.0 & $12.46^{+0.05}_{-0.4}$ \\
    SDSS J114556.26+110018.4 & $11.19^{+0.002}_{-0.9}$ \\
    SDSS J115436.60+030006.3 & $12.19^{+0.03}_{-1.1}$ \\
    SDSS J120627.62+002335.3 & $11.46^{+0.5}_{-0.6}$ \\
    SDSS J123549.95+013252.6 & $12.43^{+0.03}_{-0.4}$ \\
    SDSS J132401.53+032020.6 & $11.73^{+0.3}_{-0.9}$ \\
    SDSS J210712.77+005439.4 & $11.86^{+0.2}_{-1.6}$ \\

  \end{tabular}
\end{table}

We plotted the luminosities in \cite{farrah2012} against the luminosities we calculated to determine how 
different our values are, shown in Fig.~\ref{fig:starlum}. 

\begin{figure}
  \centering
  \includegraphics[width=\linewidth]{figs/starburst_lums.png}
  \caption{Log of the integrated IR starburst luminosity from \cite{farrah2012} plotted against ours calculated with \texttt{emcee}. 
           The dashed blue line is a one-to-one line, and the triangular caps on some error bars represent upper limits.}
           \label{fig:starlum}
\end{figure}

Preliminary interpretations suggest that many of our objects can be adequately fit with starburst components that 
have lower amplitudes. This raises the question of whether or not the quasar and starburst components are covariant. 
Fig.~\ref{fig:lyuvsfarrah_torus} shows an overlay of the quasar component used to fit SDSS J100605.66+051349.0 in \cite{farrah2012} 
and the component fit to the same object in our \texttt{emcee} analysis. The key difference between these two components is that 
the Farrah component begins to drop off at a shorter wavelength than the Lyu component. Given that this is a logarithmic plot, 
the difference between the luminosities of the two components is therefore quite large. As is clear from Fig.~\ref{fig:comps}, 
the starburst and quasar components cover the same long wavelength region, therefore they may be covariant. 
Fig.~\ref{fig:lyuvsfarrah_starburst} shows the starburst component used to fit the same object in \cite{farrah2012} overlayed 
with the starburst component from \cite{dalehelou2002} used in our \texttt{emcee} analysis, as well as the photometry for 
SDSS J100605.66+051349.0. 

When comparing these two figures, we see that the Farrah starburst component compensates for 
the low amplitude of the quasar component at longer wavelengths, such that the fit constrains the 160 $\mu$m MIPS 
photometry. However, our \texttt{emcee} results show that through using the \cite{lyu2017feb} model, the MIPS 
photometry can be constrained by a lower amplitude starburst component. This leads us to suggest that the SED fit 
is model dependent, resulting in varying starburst luminosity calculations depending on which models are used, especially 
the quasar model. Therefore \cite{farrah2012} calculates overall higher starburst luminosities than our results 
find, which can be seen in several objects, and explains the tendency for the data in Fig.~\ref{fig:starlum} to 
lie below the one-to-one line.

\begin{figure}
  \centering
  \includegraphics[width=\linewidth]{figs/lyu_vs_farrah_torus.png}
  \caption{Overlay of \cite{farrah2012} quasar component and, in blue, and \cite{lyu2017feb} quasar component, in orange. 
           Amplitude of the Lyu component is calculated with our \texttt{emcee} analysis, while the Farrah component 
           comes from a library of quasar models. Components correspond to the fit of SDSS J100605.66+051349.0}
           \label{fig:lyuvsfarrah_torus}
\end{figure}

\begin{figure}
  \centering
  \includegraphics[width=\linewidth]{figs/lyu_vs_farrah_starburst.png}
  \caption{Overlay of \cite{farrah2012} starburst component and, in blue, and \cite{dalehelou2002} starburst component, in orange. 
           Amplitude of the Dale and Helou component is calculated with our \texttt{emcee} analysis, while the Farrah component 
           comes from a library of starburst models. SDSS J100605.66+051349.0 photometry and errors in black.}
           \label{fig:lyuvsfarrah_starburst}
\end{figure}

\subsection{Interpretation and additional steps}
\label{sec:interp}

The goal of this project is to constrain the reddening and starburst components of several quasar SEDs in an effort to determine 
whether or not a correlation between star formation and high-velocity outflow exists. At this point, we have calculated the SMC 
reddening parameter and the starburst contribution to the IR luminosity for 16 objects. 

Some objects were more difficult to fit than others, as can be seen in Figs.~\ref{fig:fit1} and \ref{fig:fit2}. This was likely a 
result of low quality 
and severely reddened data, manifesting especially in the shorter wavelength regime. Given that the starburst component 
modeled the longer wavelength regime, this component was not particularly challenging to constrain. Therefore the preliminary 
interpretations of these fits are likely not as problematic as the reddening calculations. However, the quasar component 
does cover the entire wavelength range, thus any challenges in constraining the fit toward shorter wavelengths most 
likely continue into the long wavelength regime. While this would lead us to worry that there may be covariance between 
our quasar and starburst components, as discussed in \S~\ref{sec:starlum}, we can tentatively conclude that our models 
exhibit little to no covariance. Regardless, an important next step would be to focus more closely on these poorly constrained 
fits and investigate the source.
  
We also plan to reduce the additional photometry mentioned in \S~\ref{sec:data} to augment our current sample. Once we do 
this we can model all 25 objects in our sample, allowing us to more efficiently constrain a possible relation between 
outflow strength and star formation. 

To investigate whether \cite{farrah2012} is correct in claiming that outflow strength and the star formation contribution to the IR 
luminosity are anticorrelated, we need to compare the results from this paper with our separate calculations of outflow strength 
through fitting the SDSS spectra (Leighly et al. in prep). Once all problems are taken care of with this analysis and all 25 objects have been 
analyzed, we will determine if any relation exists.

\section{CONCLUSIONS}
\label{sec:conclusions}

 We have made progress in fitting the spectral energy distribution of SDSS J084044.41+363327.8, with more objects to come. In the future 
 we plan to obtain additional photometry and incorporate the data into our objects to yield a more precise constraint on the star 
 formation and reddening parameter. At this point, \texttt{emcee} is proving to be a useful tool for fitting the SEDs as it exhibits 
 fast and strong convergence despite the flaws in the model. It has shown to perform well near longer wavelengths where our model is 
 more accurate, therefore we believe it will continue to be useful as we improve our model into the future.

\section{ACKNOWLEDGMENTS}
\label{sec:acknowledgments}

A portion of this research was funded by NSF Astronomy and Astrophysics Grant \#1518382.

\newpage
\bibliography{paper}

\end{document}
